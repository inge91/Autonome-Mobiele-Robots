\documentclass[a4paper, 20pt]{article}
\usepackage[english]{babel}
\usepackage{amsmath}
\usepackage{graphicx}
\usepackage{subcaption}
\usepackage{placeins}
\usepackage[T1]{fontenc}
\usepackage[utf8]{inputenc}
\usepackage{floatrow}

\author{Maarten de Jonge \\
    Inge Becht}
\date{\today}
\title{Assignment 4\\ 
Slammin' and Jammin'}

\begin{document}
\maketitle
\section{Introduction}
In this exercise the use of the SLAM algorithm is explored for the NXT
legorobot. the maps are made using no a priori knowledge of the environment.

To be able to use SLAM, the robot needs to aqcuire landmarks from the
environment. These landsmarks in this case are wall corners. To extract these
corners the same algorithm is used as in previous exercises (2,3 and 4). 

To be able to map the information found by the sensors correctly, the robot also
needs odometry to measure its distance. This idea has been explored in a
previous exercise as well (see assignment 1), but


\section{Experiments}
Unfortunately we were unable to use the dataset recorded on the Mindstorms
robot, so all experiments were done on the provided sample logfile, in which the
robot moves forward for a bit through a hallway before turning left and
stopping.

\end{document}
