\documentclass[a4paper, 20pt]{report}
\usepackage[english]{babel}
\usepackage{amsmath}
\usepackage{graphicx}
\usepackage{subcaption}
\usepackage{placeins}

\author{Maarten de Jonge \\
Inge Becht}
\date{\today}
\title{Autonome mobiele robots}

\begin{document}
\maketitle

\begin{enumerate}
    \item For the differential drive robot:\\
        We want to find the $\xi_I$ after each chosen $\Delta t$.
        To do this we can use the formula:
        \begin{align*}
            \dot{\xi_I} = R(q)^{-1}J_1^{-1} *J_2*\varphi
        \end{align*}
        where in the case of the differential drive:
        \begin{align*}
            R = \begin{bmatrix}
                cos(q) & sin(q) & 0           \\[0.3em]
                -sin(q) & cos(q)& 0           \\[0.3em]
                0 & 0 & 1\\
            \end{bmatrix}
            J_1 =
            \begin{bmatrix}
                sin(\alpha_1 + \beta_1)  &-cos(\alpha_1 + \beta_1) & (-l)cos(\beta_1)\\
                sin(\alpha_2 + \beta_2)  &-cos(\alpha_2 + \beta_2) & (-l)cos(\beta_2)\\
                cos(\alpha_1 + \beta_1) & sin(\alpha + \beta)    &   lsin(\beta) \\
            \end{bmatrix}
        \end{align*}
        \begin{align*}
            J2 = 
            \begin{bmatrix}
                r_1 & 0 & 0 \\
                0 & r_2 & 0 \\
                0 & 0 & 0\\
            \end{bmatrix},
            \varphi(\Delta t) = 
            \begin{bmatrix}
                \varphi_1(\Delta t) \\
                \varphi_2(\Delta t )\\ 
                0\\
            \end{bmatrix}
        \end{align*}
        The values inserted in $J_1$ are those of the constraints, with the
        first two rows being the rolling constraint for both wheels and the
        third column the sliding constraint for the robot.
        The value for $\alpha_1 = \frac{-\pi}{2}  $ and $\beta_1 =\pi$. For $\alpha_2
        = \frac{\pi}{2}$. 
        $\beta_2 = 0$ for the differential drive. $J_2$ consists of the radii of
        the wheel.
        So filling this in will give: 
        \begin{align*}
            \dot{\xi_I} = R(q)^{-1} 
            \begin{bmatrix}
                0.5 & 0.5 & 0 \\
                0   & 0   & 1 \\
                \frac{1}{2l} & \frac{1}{2l} &0\\
            \end{bmatrix}
            \begin{bmatrix}
                r_1\dot{\varphi_1} \\
                r_2\dot{\varphi_2} \\
                0\\
            \end{bmatrix}
        \end{align*}

        Because with this formula only the velocity in the $x$, $y$ and $q$
        position is given the new position (at time $t+\Delta t$) can be calculated with:
        \begin{align*}
            \xi^{t + \Delta t}_I = \xi^{t}_I + (\dot{\xi^{t + \Delta t}_I} *\Delta t )
        \end{align*}
    \item
    \begin{enumerate}(a)
    \item
        \begin{align*}
            \begin{bmatrix}
            x\\
            y\\
            q\\
        \end{bmatrix}^{t + \Delta t}
            = 
            \begin{bmatrix}
            x\\
            y\\
            q\\
        \end{bmatrix}^{t}
        +( R(q)^{-1}  
            \begin{bmatrix}
                \frac{r_1\dot{\varphi_1}}{2} + \frac{r_2\dot{\varphi_2}}{2}\\
            0\\
           \frac{r_1\dot{\varphi_1}}{2l} + -\frac{r_2\dot{\varphi_2}}{2l}\\ 
       \end{bmatrix}^{t}) * \Delta t
        \end{align*}
    \item
        $ w(t) = \large{\frac{r_1\dot{\varphi_1}(t)}{2l} +
        -\frac{r_2\dot{\varphi_2}(t)}{2l}}$\\
        $ v = \frac{r_1\dot{\varphi_1}}{2} + \frac{r_2\dot{\varphi_2}}{2}$
    \end{enumerate}
\end{enumerate}


\end{document}
