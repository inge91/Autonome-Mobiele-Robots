\documentclass[a4paper, 20pt]{article}
\usepackage[english]{babel}
\usepackage{amsmath}
\usepackage{graphicx}
\usepackage{subcaption}
\usepackage{placeins}
\usepackage[T1]{fontenc}
\usepackage[utf8]{inputenc}
\usepackage{floatrow}

\author{Maarten de Jonge \\
    Inge Becht}
\date{\today}
\title{Assignment 2\\ 
Making a range finder using an omni-directional camera }

\begin{document}
\maketitle

This assignment is about making a range finder using an omni-directional
camera on a lego robot. 
Because most of the code was already given in the assignment, this report
will mostly consist of the experimenting with values for different variables to
create the best possible mapping of the camera visualisation and the real world.
In the end the error will be visualised using the best possible calibration
settings and variable values found.

\section{The steps towards simulating a range finder}

The first step towards creating the final map is calibrating the camera. For
this purpose thei script \texttt{calibrate\_camera\_offline.m} was modified to
work with a single picture taken from the omnidirectional camera, as shown in
figure \ref{fig:snapshot}


\begin{figure}[!ht]
\centering
  \includegraphics[width=0.5\textwidth]{omni_snapshot.jpg}
  \label{fig:snapshot}
  \caption{A snapshot from the omnidirectional camera}
\end{figure}

calibration values were set: $Rmin = 77,Rmax = 160, Centre\_x = 325.2203 and
Centre\_y =  225.0$

This created the spaces as can be seen in figure \ref{fig:circles}, where the space spanned between the
outer pink circle to
the inner pink circle is the used data in the rest of the report. Figure
\ref{fig:straight} shows this part of the images straightened out.

\begin{figure}[!ht]
\centering
\begin{floatrow}
  \ffigbox[\FBwidth]{\caption{Circle indicating used mirror space}\label{fig:circles}}{
  \includegraphics[width=0.5\textwidth]{map_images/camera_spaces.jpg}}
  
  \ffigbox[\FBwidth]{\caption{Flipped and straightened image} \label{fig:straight}}{
  \includegraphics[width=0.5\textwidth]{map_images/straightened_image.jpg}}
\end{floatrow}
\end{figure}


To extract the walls from the original image, it needs to be unwarped and all
data that does not fall between the two pink lines in \ref{fig:circles} needs to
be thrown away. The premade \texttt{imunwrap.m} script does exactly this. It
takes a new parameter $angstep$ which determines the angular resolution of the
simulated laser scanner. The bigger the angle the worse the resolution is, as
can be seen in figure ref{fig:angstep0.05} and ref{fig:angstep5}

\begin{figure}[!ht]
\centering
\begin{floatrow}
    \ffigbox[\FBwidth]{\caption{\texttt{imunwrap.m} applied with angstep = 0.05}\label{fig:angstep0.05}}{
  \includegraphics[width=0.5\textwidth]{rectangle_0_05.png}}
  
  \ffigbox[\FBwidth]{\caption{\texttt{imunwrap.m} applied with angstep =5}
  \label{fig:angstep5}}{
  \includegraphics[width=0.5\textwidth]{rectangle_5png.png}}
\end{floatrow}
\end{figure}

At first one would say a lower angle would be more precise in the funal mapping, 
but the problem with a lower angle is the great amount of detail that will be
kept in the image and thus a
great amount of noise that comes with it. A too high angular step, however,
could make the distance estimation less accurate.
$Angstep$ is thus one of the parameters with
wich will be experimentated to find the approximate best one.

Whatever the chosen angstep is, there still needs to be some more noise
reduction to accentuate the position of the wall pieces. A simple black and
white threshold is used for this. A more difficult choice is however at what
value to set the threshold, something that can not be determined from a single
image, but we still try to approximate it that way. In figure \ref{fig:thresh100} and
\ref{fig:thresh75} two different tresholds can be seen. Note that the white
values  of the pixels have not been normalised.
thresholded.

\begin{figure}[!ht]
\centering
\begin{floatrow}
    \ffigbox[\FBwidth]{\caption{Threshold = 100 applied with angstep =
    1}\label{fig:thresh100}}{
  \includegraphics[width=0.5\textwidth]{black_thresh_100.png}}
  
  \ffigbox[\FBwidth]{\caption{Threshold = 75. Applied with angstep =1}
  \label{fig:thresh75}}{
  \includegraphics[width=0.5\textwidth]{black_thresh_75.png}}
\end{floatrow}
\end{figure}

With this constructed image the following step is to iterate through all the
white pixels from the left side.  Now using
the height of the robot(around 0.33 meters) and a certain $\alpha$ the distance towards these
black points can be calculated.
Different values for $\alpha$ were experimented with. In case of the robot in
Zurich, $\alpha$ was given the value of 95 pixels, so we experiment around these
values as well.

\subsection{Varying parameter values}
The data set with all experimentations of angstep, the black and white
threshold and  $\alpha$ was created by running \texttt{create\_dataset.m} and can be run with
or without calibration. The following different combination of values were
used:\\

\begin{align*}
    angstep  \in \{0.05, 0.5, 1, 1.5, 2, 5, 10\}\\
    BWthresh \in \{75, 80, 85, 90, 95 100\}\\
    \alpha   \in \{80, 95, 100, 120, 130, 150\}\\
\end{align*}

\subsection{Varying $angstep$}

When using $BWthresh = 75 $ and $\alpha = 130$ and varying $angstep$, we get
figure \ref{fig:angstep1}, \ref{fig:angstep2}, \ref{fig:angstep3},
\ref{fig:angstep4}. Although first concerns were given for taking a too high
resolution, from this data we can see clearly that taking the smallest
angular step tested does not create too much noise for the purpose of line
detection. In case of the $angstep = 0.05$, we can identify 5 clear outliers and
in the case of $angstep =2$ only 3 less than this. It seems there doesn't have to
be a big trade-off between resolution and accuracy.
\begin{figure}[!ht]
\centering
\begin{floatrow}
  \ffigbox[\FBwidth]{\caption{$angstep = 0.05$ }\label{fig:angstep1}}{
  \includegraphics[width=0.7\textwidth]{fig5.jpg}}
  \ffigbox[\FBwidth]{\caption{$angstep = 0.5 $}
  \label{fig:angstep2}}{
  \includegraphics[width=0.7\textwidth]{fig41.jpg}}
\end{floatrow}
\end{figure}

\begin{figure}[!ht]
\centering
\begin{floatrow}
    \ffigbox[\FBwidth]{\caption{$angstep =1 $ }\label{fig:angstep3}}{
  \includegraphics[width=0.7\textwidth]{fig77.jpg}}
  
  \ffigbox[\FBwidth]{\caption{$angstep = 2 $}
  \label{fig:angstep4}}{
  \includegraphics[width=0.7\textwidth]{fig149.jpg}}
\end{floatrow}
\end{figure}

\FloatBarrier
\subsection{Varying $BWThresh$}
When varying the values of $BWthresh$ and keeping $angstep = 0.05$ and $\alpha=
130$ we get the results seen in figure \ref{fig:bw1}, \ref{fig:bw2},
\ref{fig:bw3} and \ref{fig:bw4}. It seems that in the chosen threshold range it
does not really matter that much what value you use.
But as stated earlier, this result should be taken lightly, as it is only
constructed with a single image with only one kind of lighting condition
\begin{figure}[!ht]
\centering
\begin{floatrow}
  \ffigbox[\FBwidth]{\caption{$BWthresh = 75$ }\label{fig:bw1}}{
  \includegraphics[width=0.7\textwidth]{fig5.jpg}}
  \ffigbox[\FBwidth]{\caption{$BWthresh = 85 $}
  \label{fig:bw2}}{
  \includegraphics[width=0.7\textwidth]{fig17.jpg}}
\end{floatrow}
\end{figure}

\begin{figure}[!ht]
\centering
\begin{floatrow}
  \ffigbox[\FBwidth]{\caption{$BWthresh = 90$ }\label{fig:bw3}}{
  \includegraphics[width=0.7\textwidth]{fig5.jpg}}
  \ffigbox[\FBwidth]{\caption{$BWthresh = 100 $}
  \label{fig:bw4}}{
  \includegraphics[width=0.7\textwidth]{fig35.jpg}}
\end{floatrow}
\end{figure}

\FloatBarrier
\subsection{Varying $\alpha$}
When varying the values of $\alpha$ and keeping $angstep = 0.05$ and $BWthresh=
80$ we get the results seen in figure \ref{fig:alpha1}, \ref{fig:alpha2}
\ref{fig:alpha3}, \ref{fig:alpha4}.These images show that $\alpha$ should not be
taken higher than 130 pixels as than the lines begin to deform in the shape of
the mirror, which has probably to do with a less than optimal calibration. This
would make it harder to do edge ore line detection. Also,
the parameter should not be chosen less than 120 pixels as than the position
estiation is not all that precise.

\begin{figure}[!ht]
\centering
\begin{floatrow}
  \ffigbox[\FBwidth]{\caption{$\alpha = 95$ }\label{fig:alpha1}}{
  \includegraphics[width=0.7\textwidth]{fig8.jpg}}
  \ffigbox[\FBwidth]{\caption{$\alpha = 100 $}
  \label{fig:alpha2}}{
  \includegraphics[width=0.7\textwidth]{fig9.jpg}}
\end{floatrow}
\end{figure}

\begin{figure}[!ht]
\centering
\begin{floatrow}
    \ffigbox[\FBwidth]{\caption{$\alpha =130 $ }\label{fig:alpha3}}{
  \includegraphics[width=0.7\textwidth]{fig11.jpg}}
  
  \ffigbox[\FBwidth]{\caption{$\alpha = 150 $}
  \label{fig:alpha4}}{
  \includegraphics[width=0.7\textwidth]{fig12.jpg}}
\end{floatrow}
\end{figure}

\FloatBarrier


\subsection{Calculating the error}
So now we know that the optimal value for $angstep = 0.05$, for $\alpha = 130$
and for $BWthresh$ it does not matter in the tested threshold. This can be used for calculating the
error using function \texttt{compute\_uncertainty}. The final result of this
can be seen in figure \ref{fig:error} in case of the optimal parameters
and in figure \ref{fig:error2} in case of less than optimal parameters.
We are not sure why the camera position creates such large errors in case of
figure \ref{fig:error2}. It does not have to do with camera
calibration,
because it never happens in case of good parameter choices. It probably is
a byproduct of using wrong 

the case with ...
\begin{figure}[!ht]
\centering
\begin{floatrow}
    \ffigbox[\FBwidth]{\caption{$The image showing the distance error in case of
    $\alpha = 130$, $angstep = 0.05$ and $BWthresh = 90$ }\label{fig:error}}{
  \includegraphics[width=0.7\textwidth]{error_image.png}}
  
  \ffigbox[\FBwidth]{\caption{The image showing the distance error in case
      of$\alpha = 90$, $angstep = 1$ and $BWthresh = 90$ }
  \label{fig:error2}}{
  \includegraphics[width=0.7\textwidth]{error_image2.jpg}}
\end{floatrow}
\end{figure}

\end{document}
