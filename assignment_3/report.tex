\documentclass[a4paper, 20pt]{article}
\usepackage[english]{babel}
\usepackage{amsmath}
\usepackage{graphicx}
\usepackage{subcaption}
\usepackage{placeins}
\usepackage[T1]{fontenc}
\usepackage[utf8]{inputenc}
\usepackage{floatrow}

\author{Maarten de Jonge \\
    Inge Becht}
\date{\today}
\title{Assignment 3\\ 
Probabilistic Pose Estimation based on Topological Map}

\begin{document}
\maketitle

In this assignment it was attempted to create a topological map of 3 different
rooms using an omni camera (as was used in the previous assignment). Two
different ways of topological mapping were attempted; using wall extraction
and by using colored blobs. In the first case landmarks were determined by
finding sequences of wall parts and corridors, and in the second case the
orientation of the colored blobs were used to determine the position of the
legorobot.
These sequences, called fingerprints, could in theory be used to distinguish between
the three different rooms, as long as the fingerprints differ enough.

\section{The dataset} 
The dataset consists of multiple pictures of every room, each with a different
orientation of the robot within the room. This will create different possible
fingerprints for both the wall based topological mapping as well as the blob
based mapping. By rotating the sequences of these fingerprints it should be
clear that two fingerprints are so much alike they refer to the same room, while
reducing the noise in the fingerprints. Figure \ref{fig:exdata} shows one of the
images from the dataset.

\begin{figure}[!ht]
\centering
  \includegraphics[width=0.5\textwidth]{data_set/2012-11-26-121531.jpg}
  \label{fig:exdata}
  \caption{A picture from the dataset. Represented is room 3.} 
\end{figure}

\section{Using wall extraction for topological mapping}
In the previous assignment we were able to succesfully find data points that
represented walls of our lego construction. This same method is now used in this
assignment (see figure \ref{fig:wallpoints}). The next step towards creating the 
final fingerprints is to first  map these points to line segments. 

\begin{figure}[!ht]
\centering
  \includegraphics[width=0.5\textwidth]{Results/points_room3.jpg}
  \label{fig:wallpoints}
  \caption{A picture of all extracted wallpoints from room 3} 
\end{figure}

After these lines are found the fingerprints are automatically  fingerprints are
automatically made by the script ..  Now the likeliness of localisation still
has to be determined using these fingerprints. A way to do this is by...

\section{Using color blobs for topological mapping}
Colors need to be detected by determining the right threshold in HSL values for
all four of the colors. Because all blobs are not exactly one color bus fall in
a range of hues (due to shadow forming and light direction) all colors have a
maximum and minimum hue that is used.  
The original image gets thresholded on these values and
after that the biggest blob is extracted by using the matlab function
\texttt{bwlabel}.


\begin{figure}[!ht]
\centering
\begin{floatrow}
    \ffigbox[\FBwidth]{\caption{\texttt{imunwrap.m}Image after initial
    thresholding}\label{fig:angstep0.05}}{
    \includegraphics[width=0.5\textwidth]{Results/red_thresh_room3.jpg}}
  
  \ffigbox[\FBwidth]{\caption{\texttt{imunwrap.m} Image with final blob detected}
  \label{fig:angstep5}}{
  \includegraphics[width=0.5\textwidth]{Results/red_threshlabel_room3.jpg}}
\end{floatrow}
\end{figure}

The provided script only works with two colors at one time to create the
fingerprint, so we devide the colors used for every room. This makes the problem
of identifying a room
a lot more straightforward, as now one of the two colors can always be chosen
uniquely. This is not all that unrealistic as color blobs could be the color of
walls in the room or other keyfeatures. The script \texttt{ComputePatStringBlobs.m} gives the option to return
the fingerprint with only a value 3 for every blob without discriminating
between
the colors.
For room 3 we use colors red and yellow. For room 2 we use colors green and
yellow and for room 1 we use colors blue and green. The set of fingerprints can
be seen in ...

\end{document}
