\documentclass[a4paper, 20pt]{article}
\usepackage[english]{babel}
\usepackage{amsmath}
\usepackage{graphicx}
\usepackage{subcaption}
\usepackage{placeins}
\usepackage[T1]{fontenc}
\usepackage[utf8]{inputenc}
\usepackage{floatrow}

\author{Maarten de Jonge \\
    Inge Becht}
\date{\today}
\title{Assignment 4\\ 
Slammin' and Jammin'}

\begin{document}
\maketitle
\section{Introduction}
In this exercise the use of the fastSLAM algorithm is explored for the NXT
legorobot. The idea of this algorithm is that the robot makes a map using no a
priori knowledge of the environment,
while at the same time determining their position in the map.
To use SLAM in general, two types of data need to be extracted.
Firstly, the robot needs to aqcuire landmarks from the
environment. These landsmarks in the case of this assignment are wall corners. To extract these
corners the same algorithm is used as in previous exercises (see assignment 2,3 and 4). 

Secondly, the robot also
needs odometry to measure its distance to map the information found by the
sensors correctly. This idea has been explored in a
previous exercise as well (see assignment 1), but uses a noise distribution to
make the next position given the previous one a certaint measure of uncertainty.

This data is collected in log files, which are used in the 

\section{Experiments}
Unfortunately we were unable to use the dataset recorded on the Mindstorms
robot, so all experiments were done on the provided sample logfile, in which the
robot moves forward for a bit through a hallway before turning left and
stopping.

\end{document}
