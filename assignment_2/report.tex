\documentclass[a4paper, 20pt]{report}
\usepackage[english]{babel}
\usepackage{amsmath}
\usepackage{graphicx}
\usepackage{subcaption}
\usepackage{placeins}
\usepackage[T1]{fontenc}
\usepackage[utf8]{inputenc}
\usepackage{floatrow}

\author{Maarten de Jonge \\
    Inge Becht\\
Duncan ten Velthuis}
\date{\today}
\title{Assignment 2\\ 
Making a range finder using an omni-directional camera }

\begin{document}
\maketitle

This assignment is about making a range finder using an omni-directional
camera on a lego robot. 
Because most of the code was already given in the assignment, this report
will mostly consist of the experimenting with values for different variables to
create the best possible mapping of the camera visualisation and the real world.
In the end the error will be visualised using the best possible calibration
settings and variable values found.

\section{The steps towards simulating a range finder}

The first step towards creating the final map is calibrating the camera. For
this purpose thei script \texttt{calibrate\_camera\_offline.m} was modified to
work with a single picture taken from the omnidirectional camera, as shown in
figure \ref{fig:snapshot}


\begin{figure}[!ht]
\centering
  \includegraphics[width=0.5\textwidth]{omni_snapshot.jpg}
  \label{fig:snapshot}
  \caption{A snapshot from the omnidirectional camera}
\end{figure}

calibration values were set: $Rmin = 77,Rmax = 160, Centre\_x = 325.2203 and
Centre\_y =  225.0$

This created the spaces as can be seen in figure \ref{fig:circles}, where the space spanned between the
outer pink circle to
the inner pink circle is the used data in the rest of the report. Figure
\ref{fig:straight} shows this part of the images straightened out.

\begin{figure}[!ht]
\centering
\begin{floatrow}
  \ffigbox[\FBwidth]{\caption{Circle indicating used mirror space}\label{fig:circles}}{
  \includegraphics[width=0.5\textwidth]{map_images/camera_spaces.jpg}}
  
  \ffigbox[\FBwidth]{\caption{Flipped and straightened image} \label{fig:straight}}{
  \includegraphics[width=0.5\textwidth]{map_images/straightened_image.jpg}}
\end{floatrow}
\end{figure}


To extract the walls from the original image, it needs to be unwarped and all
data that does not fall between the two pink lines in \ref{fig:circles} needs to
be thrown away. The premade \texttt{imunwrap.m} script does exactly this. It
takes a new parameter $angstep$ which determines the angular resolution of the
simulated laser scanner. The bigger the angle the worse the resolution is, as
can be seen in figure ref{fig:angstep0.05} and ref{fig:angstep5}

\begin{figure}[!ht]
\centering
\begin{floatrow}
    \ffigbox[\FBwidth]{\caption{\texttt{imunwrap.m} applied with angstep = 0.05}\label{fig:angstep0.05}}{
  \includegraphics[width=0.5\textwidth]{rectangle_0_05.png}}
  
  \ffigbox[\FBwidth]{\caption{\texttt{imunwrap.m} applied with angstep =5}
  \label{fig:angstep5}}{
  \includegraphics[width=0.5\textwidth]{rectangle_5png.png}}
\end{floatrow}
\end{figure}

At first one would say a lower angle would be more precise in the funal mapping, 
but the problem with a lower angle is the great amount of detail that will be
kept in the image and thus a
great amount of noise that comes with it. A too high angular step, however,
could make the distance estimation less accurate.
$Angstep$ is thus one of the parameters with
wich will be experimentated to find the approximate best one.

Whatever the chosen angstep is, there still needs to be some more noise
reduction to accentuate the position of the wall pieces. A simple black and
white threshold is used for this. A more difficult choice is however at what
value to set the threshold, something that can not be determined from a single
image, but we still try to approximate it that way. In figure \ref{fig:thresh100} and
\ref{fig:thresh75} two different tresholds can be seen. Note that the white
values  of the pixels have not been normalised.
thresholded.

\begin{figure}[!ht]
\centering
\begin{floatrow}
    \ffigbox[\FBwidth]{\caption{Threshold = 100 applied with angstep =
    1}\label{fig:thresh100}}{
  \includegraphics[width=0.5\textwidth]{blach_thresh_100.png}}
  
  \ffigbox[\FBwidth]{Threshold = 75. Applied with angstep =1}
  \label{fig:thresh75}}{
  \includegraphics[width=0.5\textwidth]{black_thresh_75.png}}
\end{floatrow}
\end{figure}

Different values for $\alpha$ were experimented with. In case of the robot in
Zurich, $\alpha$ was given the value of 95 pixels. In our experimentations we
chose for $\alpha$ the values

\
The data set with all experimentations of angstep, the black and white
threshold and  $\alpha$ was created by running \texttt{create\_dataset.m} and can be run with
or without calibration. The following different combination of values were used:e

angstep
alpha

The best looking maps were picked by hand  by looking at the final created maps,  and can be seen by




\end{document}
