\documentclass[a4paper, 20pt]{report}
\usepackage[english]{babel}
\usepackage{amsmath}
\usepackage{graphicx}
\usepackage{subcaption}
\usepackage{placeins}

\author{Maarten de Jonge \\
    Inge Becht\\
Duncan ten Velthuis}
\date{\today}
\title{Autonome mobiele robots\\ 
Assignment 1}

\begin{document}
\maketitle

\begin{enumerate}
    \item For the differential drive robot:\\
        We want to find the $\xi_I$ at each $\Delta t$ interval.
        To do this we can use the formula:
        \begin{align*}
            \dot{\xi_I} = R(q)^{-1}J_1^{-1} *J_2*\varphi(t)
        \end{align*}
        where in the case of the differential drive:
        \begin{align*}
            R = \begin{bmatrix}
                cos(q) & sin(q) & 0           \\[0.3em]
                -sin(q) & cos(q)& 0           \\[0.3em]
                0 & 0 & 1\\
            \end{bmatrix}
            J_1 =
            \begin{bmatrix}
                sin(\alpha_1 + \beta_1)  &-cos(\alpha_1 + \beta_1) & (-l)cos(\beta_1)\\
                sin(\alpha_2 + \beta_2)  &-cos(\alpha_2 + \beta_2) & (-l)cos(\beta_2)\\
                cos(\alpha_1 + \beta_1) & sin(\alpha + \beta)    &   lsin(\beta) \\
            \end{bmatrix}
        \end{align*}
        \begin{align*}
            J2 = 
            \begin{bmatrix}
                r_1 & 0 & 0 \\
                0 & r_2 & 0 \\
                0 & 0 & 0\\
            \end{bmatrix},
            \varphi(t) = 
            \begin{bmatrix}
                \varphi_1(t) \\
                \varphi_2(t )\\ 
                0\\
            \end{bmatrix}
        \end{align*}
        The values inserted in $J_1$ are those of the constraints, with the
        first two rows being the rolling constraint for both wheels and the
        third column the sliding constraint for the robot.
        The value for $\alpha_1 = \frac{-\pi}{2}  $ and $\beta_1 =\pi$. For $\alpha_2
        = \frac{\pi}{2}$ and
        $\beta_2 = 0$. $J_2$ consists of the radii of
        the wheels on the diagonal.
        Filling in these values gives:
        \begin{align*}
            \dot{\xi_I} = R(q)^{-1} 
            \begin{bmatrix}
                0.5 & 0.5 & 0 \\
                0   & 0   & 1 \\
                \frac{1}{2l} & \frac{1}{2l} &0\\
            \end{bmatrix}
            \begin{bmatrix}
                r_1\varphi_1(t) \\
                r_2\varphi_2(t) \\
                0\\
            \end{bmatrix}
        \end{align*}

        Because with this formula only the velocity in the $x$, $y$ and $q$
        direction is given the new position (at time $t+\Delta t$) can be calculated with:
        \begin{align*}
            \xi^{t + \Delta t}_I = \xi^{t}_I + (\dot{\xi}^{t}_I *\Delta t )
        \end{align*}
    \item
    \begin{enumerate}{a} 
    \item Given without the constraints:
                \begin{align*}
                    \begin{bmatrix}
                        x\\
                        y\\
                        q\\
                    \end{bmatrix}^{t + \Delta t}
                    = 
                    \begin{bmatrix}
                        x\\
                        y\\
                        q\\
                    \end{bmatrix}^{t}
                    +( R(q)^{-1}  
                    \begin{bmatrix}
                        \frac{r_1\dot{\varphi_1}}{2} + \frac{r_2\dot{\varphi_2}}{2}\\
                        0\\
                        \frac{r_1\dot{\varphi_1}}{2l} + -\frac{r_2\dot{\varphi_2}}{2l}\\ 
                    \end{bmatrix}^{t}) * \Delta t
                \end{align*}
        \item
            $ \dot{w}(t) = \large{\frac{r_1\dot{\varphi_1}(t)}{2l} +
                -\frac{r_2\dot{\varphi_2}(t)}{2l}}$\\
                $ \dot{v}(t) = \frac{r_1\dot{\varphi_1}(t)}{2} +
                \frac{r_2\dot{\varphi_2}(t)}{2}$
        \end{enumerate}

    \item
            This exercise was made in Python using the nxt-python library. To
            control the robot we created a simple movement API (see \texttt{movement.py})
            in which we distinguish between two possible motions; moving forward
            and rotating. For moving forward, the motor power and wheel rotation
            (in degrees) have to be specified, and the robot will move by
            rotating both wheels at the given rotation angle. 
            
            There's two ways to do a rotation; leaving one wheel stationary and
            rotating the other one (thus rotating around the stationary wheel),
            or spinning both wheels in the opposite direction, rotating around
            the point between the two wheels. In practise, the first method
            doesn't perform as well due to increased friction on the stationary
            wheel. The rotation function in our API takes a rotation for the
            robot to perform along with motor power. Positive motor power will
            turn counterclockwise, negative power will turn clockwise.

            To construct a path for the robot to drive, a list of forward
            movements and rotations can be specified, which will be executed
            sequentially.

            For both movement types (rotation and straight ahead) we tested both
            braking and not breaking after each movement.  brakes and not using
            brakes.  Using brakes works much better than not using brakes for
            the odometrie model as the specified number of rotations will be
            closer to the retrieved number of rotations from the sensor data.

            We created two odometry models. The first model checked after every
            movement of the robot (so after each transition from moving straight
            ahead and rotating) if the specified amount of driven milimeters was the
            same as the actually driven amount of milimeters. This is done by
            reading the rotation count from the wheels after every movement. See
            \texttt{odometrie\_tester.py} for the functions that enable this
            odometrie model and in function \texttt{path} of
            \texttt{movement.py} the interaction between movement modules and
            the rotation counts is shown. To apply this odometrie run
            \texttt{movement.py}
            
            The file \texttt{odometrie\_position\_tester.py} is the second
            odometrie model in which the kinematics model is constructed in the
            same way as is given in question 1. The idea here is to keep track
            of the robot's position ($x$, $y$ and $q$) in a global reference
            frame with a given sampling rate, taking the initial pose of the
            robot as the origin. This odometrie model starts when running
            \texttt{movement\_position\_odometrie.py} and works as a background
            thread, every 100 milliseconds updating the environmental position.

            A simple PID controller has been implemented for correcting the
            robot's trajectory when driving straight. It works by measuring the
            rotational speeds of both wheels and taking their difference as the
            error function (the more equal the wheel speeds are, the straighter
            you're driving). The nxt-python library only offers one way to
            influence the power of already rotating wheels, which wouldn't work
            well together with our earlier motion functions. For this reason the
            PID controller is implemented in a seperate test-function (in the file
            \texttt{pid.py}). It actuates by keeping the power of
            the left wheel constant (at 80), and converting the output of the
            controller to a value that will be added to the power of the right
            wheel, in an attempt to balance the wheels' powers such that they
            both rotate equally fast.
\end{enumerate}


\end{document}
